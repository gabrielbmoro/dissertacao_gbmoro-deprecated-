\documentclass[conference,letter,10pt,final]{IEEEtran}
\usepackage[utf8]{inputenc}
\usepackage[T1]{fontenc}
\usepackage{fixltx2e}
\usepackage{graphicx}
\usepackage{grffile}
\usepackage{longtable}
\usepackage{wrapfig}
\usepackage{rotating}
\usepackage[normalem]{ulem}
\usepackage{amsmath}
\usepackage{textcomp}
\usepackage{amssymb}
\usepackage{capt-of}
\usepackage{hyperref}
\usepackage[utf8]{inputenc}
\usepackage[T1]{fontenc}
\usepackage{lipsum}
\usepackage{color}
\usepackage{xspace}
\usepackage[brazil, ]{babel}
\newcommand{\review}[1]{\textcolor[rgb]{1,0,0}{[Lucas: #1]}}
\date{}
\title{Measuring Hardware Counters  for HPC Application Phase Detection}
\hypersetup{
 pdfauthor={Gabriel Bronzatti Moro, Lucas Mello Schnorr},
 pdftitle={Measuring Hardware Counters  for HPC Application Phase Detection},
 pdfkeywords={},
 pdfsubject={},
 pdfcreator={Emacs 24.3.1 (Org mode 8.3.5)}, 
 pdflang={Pt-Br}}
\begin{document}


\title{Plano de Ensino e Pesquisa}

\author{
\IEEEauthorblockN{Aluno: Gabriel Bronzatti Moro \\ Orientador: Lucas Mello Schnorr \\}
\IEEEauthorblockA{Instituto de Informática, Universidade Federal do Rio Grande do Sul \\
Caixa Postal 15064 –- CEP 91501-970 Porto Alegre -- RS -- Brasil\\
Email: \textit{\{gabriel.bmoro,schnorr\}@inf.ufrgs.br}\\
}
}

\maketitle

\begin{abstract}

\end{abstract}

\section{Introduction}
\label{sec:orgheadline1}
\section{Estado da Arte}
\label{sec:orgheadline2}

Laurenzano et al. 2011\cite{laurenzano2011reducing} apresentam uma abordagem automatizada que permite selecionar a frequência mais adequada de processador para determinado laço do programa. A frequência do processador é escolhida utilizando como base uma análise estática (realizada antes da execução) e outra análise realizada durante o tempo de execução da aplicação, utilizando os rastros obtidos. Os autores utilizaram vários \textit{benchmarks}, tendo como base de execução o framework chamado pcubed (\textit{PMaC's Performance and Power benchmark}) que permite explorar diferentes comportamentos de laços de interações a fim de definir uma caracterização para a máquina alvo. A caracterização da máquina define valores como consumo de potência, desempenho, padrões de execução e frequências de processador. Os resultados obtidos no experimento podem ser utilizados posteriormente como base de conhecimento, assim é possível visualizar o comportamento do consumo de energia quando se ajusta os fatores de caracterização da máquina. Dentre os resultados obtidos pelo trabalho, o melhor foi a redução de até 10,6$\backslash$% no consumo de energia.


\begin{table}[h]
\centering
\caption{Principais pontos sobre o trabalho Laurenzano et al. 2011 \cite{laurenzano2011reducing}.}
\begin{tabular}{|l|l|}
\hline
\textbf{Aplicação} & Sequenciais e Paralelas \\
\hline
\textbf{Região} & Laço\\
\hline
\textbf{Medidas} & Taxa de misses da L1 e L2\\
& Operações sobre inteiros  \\
& Operações de Ponto-flutuante  \\
\hline
\textbf{Métricas} & Power Delay Product \\
\hline
\textbf{Técnica} & Instrument. de Binário \\
\hline
\end{tabular}
\end{table}


\bibliographystyle{IEEEtran}
\bibliography{refs}
\end{document}
