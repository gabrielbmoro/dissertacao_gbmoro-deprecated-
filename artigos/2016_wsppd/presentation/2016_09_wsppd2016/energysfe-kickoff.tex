% Created 2016-09-01 Qui 14:33
\documentclass[12pt,xcolor=dvipsnames,presentation,handout]{beamer}
\usepackage[utf8]{inputenc}
\usepackage[T1]{fontenc}
\usepackage{fixltx2e}
\usepackage{graphicx}
\usepackage{grffile}
\usepackage{longtable}
\usepackage{wrapfig}
\usepackage{rotating}
\usepackage[normalem]{ulem}
\usepackage{amsmath}
\usepackage{textcomp}
\usepackage{amssymb}
\usepackage{capt-of}
\usepackage{hyperref}
\graphicspath{{../}}
\input{../org-babel-style-preembule.tex}
\institute{
\includegraphics[width=.16\textwidth]{img/gppd.png}
\hfill
\includegraphics[width=.16\textwidth]{img/inf.pdf}
\hfill
\includegraphics[width=.16\textwidth]{img/ufrgs.pdf}
% \hfill
% \includegraphics[width=.16\textwidth]{img/cnpq.pdf}
\hfill
\includegraphics[width=.18\textwidth]{img/hpe.png}
}
\usetheme{default}
\author{Gabriel B. Moro and Lucas M. Schnorr \\ \{gbmoro,schnorr\}@inf.ufrgs.br}
\date{XIV Workshop de Processamento Paralelo e Distribuído \linebreak UFRGS, Porto Alegre, 2nd September 2016}
\title{Measuring Hardware Counters for HPC Application Phase Detection}
\hypersetup{
 pdfauthor={Gabriel B. Moro and Lucas M. Schnorr \\ \{gbmoro,schnorr\}@inf.ufrgs.br},
 pdftitle={Measuring Hardware Counters for HPC Application Phase Detection},
 pdfkeywords={},
 pdfsubject={},
 pdfcreator={Emacs 24.3.1 (Org mode 8.3.5)}, 
 pdflang={English}}
\begin{document}

\maketitle
\input{../org-babel-document-preembule.tex}
\newcommand{\prettysmall}[1]{\fontsize{#1}{#1}\selectfont}

\section{Introduction}
\label{sec:orgheadline3}
\subsection{}
\label{sec:orgheadline2}
\begin{frame}[fragile,label={sec:orgheadline1}]{Introduction}
 \texttt{Motivation}:

\begin{itemize}
\item Reducing the power consumption of parallel applications;
\item Fine-granularity to identify the \uline{memory-bound regions} of parallel
application several timestamps;
\item \uline{Lower overhead} of measurement.
\end{itemize}

\texttt{Objective}:

\begin{itemize}
\item Measure hardware counters at every given time interval to discover
memory-bound regions
\end{itemize}
\end{frame}

\section{Related Works}
\label{sec:orgheadline7}
\subsection{}
\label{sec:orgheadline6}
\begin{frame}[fragile,label={sec:orgheadline4}]{Related Works}
 \texttt{Spiliopoulos et al}:
\begin{itemize}
\item Tool that analyzes the behavior of sequential application (the
concept of phases);
\item Based on cache misses of different caches' levels.
\end{itemize}

\texttt{Laurenzano et al}: 
\begin{itemize}
\item Finer granularity for each application loop.
\end{itemize}

\texttt{Freeh et al.}:
\begin{itemize}
\item Define the most suitable frequency for each phase of MPI
applications;
\item Analyse of the best frequency for each node.
\end{itemize}
\end{frame}


\begin{frame}[label={sec:orgheadline5}]{References}
\tiny
\bibliographystyle{plain}
\bibliography{europar2016}
\end{frame}
\end{document}
