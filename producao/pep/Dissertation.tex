\documentclass[cic,tc,english]{iiufrgs}
\usepackage[utf8]{inputenc}
\usepackage[T1]{fontenc}
\usepackage{fixltx2e}
\usepackage{graphicx}
\usepackage{grffile}
\usepackage{longtable}
\usepackage{wrapfig}
\usepackage{rotating}
\usepackage[normalem]{ulem}
\usepackage{amsmath}
\usepackage{textcomp}
\usepackage{amssymb}
\usepackage{capt-of}
\usepackage{hyperref}
\usepackage[utf8]{inputenc}
\usepackage[T1]{fontenc}
\usepackage{subfigure}
\usepackage{tabulary}
\usepackage{tabularx}
\usepackage{mathtools}
\usepackage{algorithm}
\usepackage{algorithmic}
\usepackage{listings}
\newcommand{\prettysmall}{\fontsize{6.5}{6.5}\selectfont}
\newcommand{\prettysmallbis}{\fontsize{7}{7}\selectfont}
\newcommand{\mtilde}{~}
\usepackage[utf8]{inputenc}
\usepackage[T1]{fontenc}
\usepackage{palatino}
\usepackage{hyperref}
\usepackage{cleveref}
\usepackage{booktabs}
\usepackage[normalem]{ulem}
\usepackage{xspace}
\usepackage{amsmath}
\usepackage{color}
\newcommand{\review}[1]{\textcolor[rgb]{1,0,0}{[Orientador: #1]}}
\newcommand{\Orientador}[1]{\textcolor[rgb]{0.2,0.2,0.7}{[Orientador: #1]}}
\newcommand{\source}{Source: Author}
\input{configuration.tex}
\date{}
\title{}
\hypersetup{
 pdfauthor={Aluno},
 pdftitle={},
 pdfkeywords={},
 pdfsubject={},
 pdfcreator={Emacs 24.3.1 (Org mode 8.3.5)}, 
 pdflang={English}}
\begin{document}


\title{Plano de Ensino e Pesquisa}
\author{Sobrenome}{Aluno}
\advisor[Prof.~Dr.]{Sobrenome}{Orientador}

\date{Outubro}{2016}
\location{Porto Alegre}{RS}

% \renewcommand{\nominataReit}{Prof\textsuperscript{a}.~Wrana Maria Panizzi}
% \renewcommand{\nominataReitname}{Reitora}
% \renewcommand{\nominataPRE}{Prof.~Jos{\'e} Carlos Ferraz Hennemann}
% \renewcommand{\nominataPREname}{Pr{\'o}-Reitor de Ensino}
% \renewcommand{\nominataPRAPG}{Prof\textsuperscript{a}.~Joc{\'e}lia Grazia}
% \renewcommand{\nominataPRAPGname}{Pr{\'o}-Reitora Adjunta de P{\'o}s-Gradua{\c{c}}{\~a}o}
% \renewcommand{\nominataDir}{Prof.~Philippe Olivier Alexandre Navaux}
% \renewcommand{\nominataDirname}{Diretor do Instituto de Inform{\'a}tica}
% \renewcommand{\nominataCoord}{Prof.~Carlos Alberto Heuser}
% \renewcommand{\nominataCoordname}{Coordenador do PPGC}
% \renewcommand{\nominataBibchefe}{Beatriz Regina Bastos Haro}
% \renewcommand{\nominataBibchefename}{Bibliotec{\'a}ria-chefe do Instituto de Inform{\'a}tica}
% \renewcommand{\nominataChefeINA}{Prof.~Jos{\'e} Valdeni de Lima}
% \renewcommand{\nominataChefeINAname}{Chefe do \deptINA}
% \renewcommand{\nominataChefeINT}{Prof.~Leila Ribeiro}
% \renewcommand{\nominataChefeINTname}{Chefe do \deptINT}


%
% TODO: provide these keywords
%
\keyword{HPC}

\maketitle

\begin{abstract}
Abstract ...
\end{abstract}

\listoffigures
\listoftables

% lista de abreviaturas e siglas
% o parametro deve ser a abreviatura mais longa
\begin{listofabbrv}{SPMD}
   \item[ANTLR] Another Tool For Language Recognition
   \item[CSV] Comma Separated Values
   \item [DBMS] Database Management System    
   \item[GC] Garbage Collector 
   \item[HPC] High Performance Computing
   \item[JDBC] Java Database Connectivity
   \item[JVM] Java Virtual Machine
\end{listofabbrv}


% idem para a lista de símbolos
% \begin{listofsymbols}{$\alpha\beta\pi\omega$}
%     \item[$\sum{\frac{a}{b}}$] Somatório do produtório
%     \item[$\alpha\beta\pi\omega$] Fator de inconstância do resultado
% \end{listofsymbols}

\tableofcontents

\chapter{Introdução}
\label{sec:orgheadline1}

\chapter{Conceitos Básicos}
\label{sec:orgheadline2}
\label{chapter.basic_concepts}

%\begin{figure}[!htb]
%\caption{JavaCC's file generation flow}
%\centering
%\includegraphics[width=.85\linewidth]{./img/javaccex.pdf}
%\label{fig.javaccex} 
%\\\source
%\end{figure}

\chapter{Trabalhos Relacionados}
\label{sec:orgheadline3}
\label{chapter.relatedwork}

Laurenzano et al. 2011\cite{laurenzano2011reducing} apresentam uma abordagem automatizada que permite selecionar a frequência mais adequada de processador para determinado laço do programa. A frequência do processador é escolhida utilizando como base uma análise estática (realizada antes da execução) e outra análise realizada durante o tempo de execução da aplicação, utilizando os rastros obtidos. Os autores utilizaram vários \textit{benchmarks}, tendo como base de execução o framework chamado pcubed (\textit{PMaC's Performance and Power benchmark}) que permite explorar diferentes comportamentos de laços de interações a fim de definir uma caracterização para a máquina alvo. A caracterização da máquina define valores como consumo de potência, desempenho, padrões de execução e frequências de processador. Os resultados obtidos no experimento podem ser utilizados posteriormente como base de conhecimento, assim é possível visualizar o comportamento do consumo de energia quando se ajusta os fatores de caracterização da máquina. Dentre os resultados obtidos pelo trabalho, o melhor foi a redução de até 10,6$\backslash$% no consumo de energia.

\chapter{Experimentos}
\label{sec:orgheadline4}
\label{chapter.experiments} 


%\begin{table}[!htb]
%\caption{Experimental Units description}
%\label{tab.machines}
%\centering
%\begin{tabularx}{\linewidth}{lXXX}\toprule
%                        &   {\bf Luiza}         & {\bf Orion1}         & {\bf Guarani}    %\\\toprule
%Processor                &  Intel Core i7        & Xeon E5-2630         & Intel Core i5-2400   %\\
%CPU(s)                   &  1                    & 2                    & 1       \\
%Cores per CPU            &  4                    & 6                    & 4             \\
%Max. Freq.               &  2.7 GHz              & 2.30GHz              & 3.10GHz       \\
%L1d/L1i Cache            & 32/32KBytes           & 32/32KBytes          & 32/32KBytes  \\     
%L2 Cache                 & 256KBytes             & 256KBytes            & 256KBytes    \\
%L3 Cache                 & 6MBytes               & 15MBytes              & 6MBytes         \\
%Memory                   & 16GBytes              & 32GBytes             & 20GBytes      %\\\midrule
%OS                       & OSX 10.10.5           & Ubuntu 12.04.5       & Debian 4.3.5-1 \\
%\bottomrule\end{tabularx}
%\end{table}

\bibliography{References}


\appendix
\chapter{Apendice 1}
\label{sec:orgheadline5}
\label{ap.javacc}

asdsaodk
set of optional flags. An example, is the flag \texttt{STATIC}, which means
that there is only one parser for the JVM when set to true.
\end{document}
